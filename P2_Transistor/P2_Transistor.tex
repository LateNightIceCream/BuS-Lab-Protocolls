\documentclass[a4paper, 12pt]{article}

\input{../preamble}

\setlength\columnsep{0.145898\paperwidth}
%%%%%%%%%%%%%%%%%%%%%%%%%%%%%%%%%%%%%

\begin{document}

% 1
%%%%%%%%%%%%%%%%%%%%%%%%%%%%%%%%%%%%%
  \includepdf{./titlepage/titlepage1.pdf}
  \clearpage
  \setcounter{page}{1}
%%%%%%%%%%%%%%%%%%%%%%%%%%%%%%%%%%%%%

\section{Vorbereitungsaufgaben}


\subsection{Aufbau und Wirkungsweise des BJT}

\begin{center}
  \includegraphics[width=\textwidth]{1_1/npn}
\end{center}

Bipolartransistoren können, ähnlich wie Dioden, durch Verkettung von insgesamt 3
dotierten Halbleiterkristallen, jeweils entweder p- oder
n-leitend, realisiert werden, woraus sich die beiden Kombinationsarten
\textit{npn} und \textit{pnp} ergeben.

\subsection{4-Quadranten-Kennlinienfeld}

\begin{center}
  \includegraphics[width=\textwidth]{1_2/2_1_4Q}
\end{center}


\noindent Quadrant IV stellt die Rückwirkung der Kollektor-Emitter-Spannung auf
die Basis-Emitter-Spannung dar, wird jedoch oft nicht verwendet, weshalb man
sich meist auf die Quadranten I, II und III beschränkt.


% 2
%%%%%%%%%%%%%%%%%%%%%%%%%%%%%%%%%%%%%
\pagebreak
\includepdf{./titlepage/titlepage2.pdf}
  \clearpage
\setcounter{page}{1}
%%%%%%%%%%%%%%%%%%%%%%%%%%%%%%%%%%%%%

\setcounter{section}{0}
\section{Versuchsaufgaben}



\end{document}
