\documentclass[a4paper, 12pt]{article}

\input{../preamble}

%%%%%%%%%%%%%%%%%%%%%%%%%%%%%%%%%%%%%

\begin{document}

%%%%%%%%%%%%%%%%%%%%%%%%%%%%%%%%%%%%%
  \includepdf{./titlepage/titlepage.pdf}
  \clearpage
  \setcounter{page}{1}
%%%%%%%%%%%%%%%%%%%%%%%%%%%%%%%%%%%%%

\section{Vorbereitungsaufgaben}

% 1.2
\subsection{Aufbau und Wirkungsweise eines pn-Übergangs}
\begin{center}
  \includegraphics[width=\textwidth]{pn}
\end{center}

\noindent Dotiert man einen Halbleiterkristall (z.B. \ce{Si} oder \ce{Ge}) mit
Fremdatomen, wird die elektrische Leitfähigkeit des Halbleiters beeinflusst.

\begin{itemize}
\item{
    Bei Dotierung mit
    5-(oder höher-)wertigen Atomen (z.B. \ce{P} oder \ce{As}) geraten zusätzliche
    Elektronen in das Leitungsband des Halbleiterkristalls; es wird \emph{n-Leitung}
    provoziert
  }

\item{
    Bei Dotierung mit
    3-(oder geringer-)wertigen Atomen (z.B. \ce{B} oder \ce{Ga}) entstehen
    Elektronenfehlstellen im Valenzband des Halbleiterkristalls; es wird \emph{p-Leitung}
    provoziert.
  }
\end{itemize}

Die Elektronenkonzentration im n-Leiter ist somit höher als die im p-Leiter.
Bringt man unterschiedlich dotierte Halbleiter in Kontakt, kommt es durch Diffusion zum Übergang von (höher-energetischen) Elektronen im Leitungsband des n-Leiters in das (nieder-energetische) Valenzband des p-Leiters. Im p-Leiter werden dann die Elektronenfehlstellen gefüllt; Im n-Leiter werden Fehlstellen von den Elektronen zurückgelassen. Die Diffusion erfolgt bis der Konzentrationsunterschied beider Halbleiter ausgeglichen ist und die Elektronenkonfigurationen der Kristalle 


% 1.1
\subsection{Aufbau und Wirkungsweise einer Diode}
Die elektrischen Eigenschaften des pn-Übergangs können technisch ausgenutzt werden, um eine \emph{Diode} zu realisieren. Der Stromfluss durch den pn-Übergang ist von der Polarität der über ihn angelegten Spannung abhängig. Man definiert daher die Orientierung der Diode in \emph{Sperr-} und \emph{Durchlassrichtung}.


% 1.3
\subsection{Z- und Schottky-Dioden}

% 1.4
\subsection{Spannungsstabilisierungsschaltung}

% 1.5
\subsection{Dimensionierung des Vorwiderstands}

% 1.6
\subsection{Diodenkennlinie}

% 1.7
\subsection{Diodengrenzwerte}


% 2
%%%%%%%%%%%%%%%%%%%%%%%%%%%%%%%%%%%%%
\section{Versuchsaufgaben}

% 2.1
\subsection{Diodenkennline BY500}

% 2.2
\subsection{Diodenkennline ZY5,6}

% 2.3
\subsection{Z-Spannung und differentieller Widerstand}

% 2.4
\subsection{Spannungsstabilisierung bei veränderlicher Eingangsspannung}

% 2.5
\subsection{Spannungsstabilisierung bei veränderlichem Laststrom}

\end{document}
