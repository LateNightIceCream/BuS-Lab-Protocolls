\documentclass[a4paper, 12pt]{article}

\input{../preamble}

%%%%%%%%%%%%%%%%%%%%%%%%%%%%%%%%%%%%%

\begin{document}

%%%%%%%%%%%%%%%%%%%%%%%%%%%%%%%%%%%%%
  \includepdf{./titlepage/titlepage.pdf}
  \clearpage
  \setcounter{page}{1}
%%%%%%%%%%%%%%%%%%%%%%%%%%%%%%%%%%%%%

\section{Vorbereitungsaufgaben}

% 1.1
\subsection{Aufbau und Wirkungsweise eines pn-Übergangs}
Dotiert man einen Halbleiterkristall (z.B. \ce{Si} oder \ce{Ge}) mit Fremdatomen, 


% 1.1
\subsection{Aufbau und Wirkungsweise einer Diode}
Die elektrischen Eigenschaften des pn-Übergangs können technisch ausgenutzt werden, um eine \emph{Diode} zu realisieren. Der Stromfluss durch den pn-Übergang ist von der Polarität der über ihn angelegten Spannung abhängig. Man definiert daher die Orientierung der Diode in \emph{Sperr-} und \emph{Durchlassrichtung}.

% 1.1
\subsection{Z- und Schottky-Dioden}

% 1.1
\subsection{Spannungsstabilisierungsschaltung}

% 1.1
\subsection{Dimensionierung des Vorwiderstands}

% 1.1
\subsection{Diodenkennlinie}

% 1.1
\subsection{Diodengrenzwerte}


% 2
%%%%%%%%%%%%%%%%%%%%%%%%%%%%%%%%%%%%%
\section{Versuchsaufgaben}

% 2.1
\subsection{Diodenkennline BY500}

% 2.2
\subsection{Diodenkennline ZY5,6}

% 2.3
\subsection{Z-Spannung und differentieller Widerstand}

% 2.4
\subsection{Spannungsstabilisierung bei veränderlicher Eingangsspannung}

% 2.5
\subsection{Spannungsstabilisierung bei veränderlichem Laststrom}

\end{document}
