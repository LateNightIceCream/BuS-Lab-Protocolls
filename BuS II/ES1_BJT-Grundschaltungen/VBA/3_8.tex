\subsubsection{Bestimmung des Eingangswiderstands}
Allgemein gelingt die Widerstandsmessung indirekt durch eine Spannungs- und
Strommessung und anschließende Division der gemessenen Größen. 

\subsubsection{Bestimmung des Ausgangswiderstands}
Der Zusammenhang zwischen Lastwiderstand und Ausgangswiderstand des Verstärkers
ist linear (Zweipol)
\[R_a = \frac{U_{a,0} - U_{a,Last}}{I_{a,Last}}\]
\[R_a = \frac{U_{a,0} - U_{a,Last}}{U_{a,Last}} \cdot R_L\]
\[R_a = R_L \cdot \left(\frac{U_{a,0}}{U_{a,Last}}-1 \right)\]

Wobei $U_{a,0}$ die Leerlaufspannung am Ausgang, $U_{a,Last}$ die
Spannung am Ausgang bei Belastung mit dem Lastwiderstand $R_L$, und $R_a$ der
Ausgangswiderstand des Verstärkers ist.

Aus der Gleichung lässt sich erkennen, dass $R_a = R_L$ wenn $U_{a,Last} =
U_{a,0}/2$ ist. D.h. die Ausgangsspannung $U_{a,Last}$ kann bei
Belastung mit einem bekannten Lastwiderstandswert gemessen und anschließend in
der obigen Gleichung verwendet werden oder sie kann auf den halben Wert der
Leerlaufausgangsspannung durch einen variablen Lastwiderstand eingestellt
werden, wonach eine Messung des Lastwiderstandes folgt.


\subsubsection{Bestimmung der Verstärkungen}
Da die Verstärkungen frequenzabhängig sind und somit das Verhältnis von Ein- und
Ausgangsspannung (wechselspannungsmäßig) nicht konstant ist, muss man messtechnisch die
Übertragungsfunktion ermitteln, was beispielsweise durch das Durchlaufen mehrerer
Signalfrequenzen und anschließender Darstellung der
Ausgangsamplituden/-differenz erreicht werden kann (wobbeln). 