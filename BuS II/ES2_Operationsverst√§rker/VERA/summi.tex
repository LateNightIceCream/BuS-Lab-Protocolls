\subsubsection{Dimensionierung}
Die untersuchte Summierschaltung arbeitete nach der Funktion
\[U_a = -(2 U_1 + 4 U_2)\]
Der Rückkoppelwiderstand war $R_3 = 20 \, \si{\kilo\ohm}$. Die Dimensionierung
erfolge durch
\[\frac{R_3}{R_1} = 2, \,\, \frac{R_3}{R_2} = 4\]
und ergab
\[R_1 = 10 \, \si{\kilo\ohm}, \,\, R_2 = 5 \, \si{\kilo\ohm}\]

\subsubsection{Messung}
% Please add the following required packages to your document preamble:
% \usepackage{booktabs}
% \usepackage[table,xcdraw]{xcolor}
% If you use beamer only pass "xcolor=table" option, i.e. \documentclass[xcolor=table]{beamer}
\begin{table}[H]
  \begin{center}
\begin{tabular}{@{}llll@{}}
\rowcolor{gray1} 
$U_1 / \si{\volt}$                                               & $0.5$     & $1$       & $1.5$     \\
\rowcolor{gray1} 
$U_2 / \si{\volt}$                                               & $0.5$     & $0.5$     & $1$       \\
\cellcolor{gray1}$U_a, \textrm{gemessen} / \si{\volt}$    & $-3.0049$ & $-4.0004$ & $-6.9967$ \\
  \cellcolor{gray1}$U_a, \textrm{theoretisch} / \si{\volt}$ & $-3$      & $-4$      & $-7$     
\end{tabular}
  \end{center}
  \caption{Messwerte der Summierschaltung}
\end{table}

Die Messwerte wurden mit einem Digitalmultimeter aufgenommen und zeigen eine sehr präzise Arbeitsweise der Schaltung.