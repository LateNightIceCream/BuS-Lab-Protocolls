Der BCD-Code ist ein Code zur Darstellung von Dezimalziffern (0-9), webei jeder
Ziffer 4-bit zugeordnet werden. Er wird auch 8-4-2-1-Code genannt, da dies der
dezimalen Wertigkeit der jeweiligen Binärstellen im Code entspricht.

\begin{table}[H]
  \begin{center}
\begin{tabular}{cc}
\rowcolor{gray0} 
Dezimalziffer & BCD-Code \\
0             & 0000     \\
1             & 0001     \\
2             & 0010     \\
3             & 0011     \\
4             & 0100     \\
5             & 0101     \\
6             & 0110     \\
7             & 0111     \\
8             & 1000     \\
9             & 1001    
\end{tabular}
\end{center}
\end{table}

Ein Dekadenzähler zählt im BCD-Code von 0 (0000) bis 9 (1001) (Reset bei 10).